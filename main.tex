\documentclass{article}
\usepackage{graphicx} % Required for inserting images
\usepackage[ngerman]{babel}

\title{Pflichtenheft \\ \large Einflussfaktoren auf die Verkehrsmittelwahl\\ -- Baukasten für Discrete Choice Modelle}
\author{Kevin Boehnke \\ \texttt{uxpkw@student.kit.edu}
\and Floriane Bresser \\ \texttt{uspvq@student.kit.edu}
\and Damian Reich \\ \texttt{uqppn@student.kit.edu}
\and Alissa Saleh \\ \texttt{unmbc@student.kit.edu}
\and Michael Schur \\ \texttt{ufkmz@student.kit.edu}}
\date{24. Mai 2023}

\begin{document}
\clearpage\maketitle\thispagestyle{empty}
\newpage
\clearpage\tableofcontents\thispagestyle{empty}
\newpage
\pagenumbering{arabic}

\section{Einleitung}

\section{Zielbestimmung}
Das Produkt soll die Verkehrsingeniure dabei unterstützen, Verkehrsmodelle (einfach oder effizient?) zu erstellen, modifizieren und visualisieren.
\subsection{Musskriterien}
\begin{itemize}
    \item /MK1/ Einlesen von CSV-Dateien mit Erhebungsdaten mit dynamischen Spaltenbezeichnungen
    \item /MK2/ Definieren von Funktionen zur Ableitung weiterer Attribute, die nicht im CSV enthalten sind.
        \subitem Intervalle
        \subitem Gruppen
        \subitem Logische Ausdrücke
        \subitem Vergleiche
    \item /MK3/ Definieren von Nutzenfunktionen für Alternativen als Linearkombination
        \subitem Unterstützung von verschachtelte Nutzfunktionen
    \item /MK4/ Signifikanzberechnung mittels einer geeigneten Bibliothek (z. B. Apollo oder biogeme)
    \item /MK5/ Graphische Benutzeroberfläche
    \item /MK6/ Nutzerarbeit(-eingaben) müssen zwischengespeichert \& geladen werden können
    \item /MK7/ Verständliche Kommunikation von regulären Interaktionen sowie Fehlern
        \subitem keine Stacktraces
        \subitem klar sprechende Fehlermeldungen
        \subitem keine hidden/silent changes
    \item /MK8/ Schnittstelle um die Parameteraufteilung unter Betracht der Signifikanz  zu automatisieren, erweitern und einzuteilen
        \subitem Erweiterbarkeit wird am Beispiel von verschiedenen Altersgruppen demonstriert
    \item /MK9/ Visualisierung der Parameterwerte und Signifikanzniveaus
        \subitem Nicht signifikante Parameter werden hervorgehoben
    \item /MK10/ Schnittstelle um andere Bibliotheken oder Implementierungen zur Parameterschätzung einzubinden
    \item /MK11/ Weitergabe der Nutzerszenarien an die Apollo Bibliothek in R
\end{itemize}

\subsection{Wunschkriterien}
\begin{itemize}
    \item Arbiträre Funktionen in der Attributaufbereitung
    \item Speichern der nutzerdefinierten Funktionen
    \item Unterstützung bei Nutzereingabe durch
        \subitem Autovervollständigung
        \subitem Typhinweise
        \subitem Fehlerunterbindung
    \item Verschachtelte Nutzenfunktionen im Modell-Aufbau
        \subitem Arithmetik: $\beta \cdot (T + (\beta_2 \cdot X_2 \cdots))$
        \subitem Exponentialfunktionen: $\log(T)$, ${\rm e}^T$
        \subitem Potenzen: $\beta \cdot X^3$
        \subitem Erweiterbar um andere Modell-Strukturen (bspw. Nested Logit) 
    \item vollwertiger Algorithmus zur Bestimmung von Signifikanzgruppen und Aufteilungen
    \item Konfigurierbare Schwellwerte für die Signifikanz
        \subitem Beispiel apollo: $|T-Ratio | > 1.95$, $|Robust T-Ratio | > 1.95$
    
\end{itemize}
\subsection{Abgrenzungskriterien}
    
\section{Produkteinsatz}
\subsection{Anwendungsbereiche}
\begin{itemize}
    \item In der Verkehrsbereich und in der Recherche
\end{itemize}
\begin{itemize}
    \item Prognose und quantitative Basis für verkehrsplanerische, (betriebswirtschaftliche) und politische Entscheidungen?
\end{itemize}
\subsection{Zielgruppen}

\begin{itemize}
    \item Verkehrsingenieure und Verkehrswissenschaftler
    \item Keine Informatiker (außer um Code zu erweitern)
\end{itemize}
  
\subsection{Betriebsbedingungen}

\section{Produktumgebung}
\subsection{Software}
\begin{itemize}
    \item Die Software wird in Python geschrieben.
    \item Betriebssystem: Windows 10 oder Windows 11
    \item \textit{WK}: Die Anwendung soll über eine Schnittstelle andere Bibliotheken zur Parameterschätzung einbinden können.
    \begin{itemize}
        \item beispielsweise \textit{R Package Apollo}
    \end{itemize}
    \item Die Software wird als Open Source veröffentlicht und erweiterbar sein.
\end{itemize}
\subsection{Hardware}
\begin{itemize}
    \item Das Produkt ist eine Desktopanwendung.
    \begin{itemize}
        \item Mindestanforderung: Laptop, 8GB RAM, Intel Core i5 (4 Kerne, 8 logische Kerne)
        \item \textit{WK}: Workstation, Windows Server, 8 Kerne (16 logische Kerne)
    \end{itemize}
\end{itemize}

\section{Produktfunktionen}
\subsection{Funktionsübersicht}

\section{Produktdaten}

\section{Produktleistungen}

\section{Benutzungsschnittstelle}

\section{Globale Testfälle}

\section{Qualitätsbestimmung}

\end{document}
