\documentclass{article}
\usepackage{graphicx} % Required for inserting images
\usepackage[ngerman]{babel}
\usepackage{enumitem}
\usepackage{float}
\usepackage{chngcntr}
\usepackage{glossaries}
\usepackage{tabularx}
\usepackage{hyperref}
\usepackage{titletoc}
\counterwithin{figure}{section}
\counterwithin{table}{section}
\setlength\parindent{0pt}

\makeglossaries

\newglossaryentry{Attributsableitung}
{
    name=Attributsableitung,
    description={Platzhalter-Implementierung eines Glossar-Eintrags}
}

\newglossaryentry{Verkehrsmodell}
{
    name=Verkehrsmodell,
    description={Platzhalter-Implementierung eines Glossar-Eintrags}
}

\newglossaryentry{Discrete Choice}
{
    name=Discrete Choice,
    description={Platzhalter-Implementierung eines Glossar-Eintrags}
}

\newglossaryentry{Projektdatei}
{
    name=Projektdatei,
    description={Enthält eine CSV-Datei, sowie eventuell Attributsableitungen, Alternativen und vorherige Ergebnisse.}
}

\newglossaryentry{Valide Attributsableitung}
{
    name=Valide Attributsableitung,
    description={Eine Attributsableitung, dessen Name eindeutig ist, die syntaktisch korrekt ist und dessen Referenzen auf Attribute alle existieren.}
}

\newglossaryentry{Invalide Attributsableitung}
{
    name=Valide Attributsableitung,
    description={Eine Attributsableitung, dessen Name bereits vorkommt, die syntaktisch inkorrekt ist oder die eine Referenz auf ein nicht-existierendes Attribut hat.}
}

\newglossaryentry{Valide Alternative}
{
    name=Valide Alternative,
    description={Eine Alternative, dessen Name eindeutig ist, dessen Nutzenfunktion syntaktisch korrekt ist und dessen Referenzen auf Attribute alle existieren.}
}

\newglossaryentry{Invalide Alternative}
{
    name=Invalide Alternative,
    description={Eine Alternative, dessen Name bereits vorkommt, dessen Nutzenfunktion syntaktisch inkorrekt ist oder die eine Referenz auf ein nicht-existierendes Attribut hat.}
}

\newglossaryentry{Alternative}
{
    name=Alternative,
    description={Ein alternatives Verkehrsmittel.}
}

\newacronym{Ableitung}{Ableitung}{Attributsableitung}

\newacronym{WK}{WK}{Wunschkriterium}

\newacronym{CSV}{CSV}{Comma-separated values}

\newacronym{JSON}{JSON}{JavaScript Object Notation}

\title{Pflichtenheft \\ \large Discrete Choice Model Builder}
\author{Kevin Boehnke \\ \texttt{uxpkw@student.kit.edu}
\and Floriane Bresser \\ \texttt{uspvq@student.kit.edu}
\and Damian Reich \\ \texttt{uqppn@student.kit.edu}
\and Alissa Saleh \\ \texttt{unmbc@student.kit.edu}
\and Michael Schur \\ \texttt{ufkmz@student.kit.edu}}
\date{26. Mai 2023}

\begin{document}
\clearpage\maketitle\thispagestyle{empty}
\newpage
\clearpage
%\tableofcontents
\startcontents[maintableofcontents]
\printcontents[maintableofcontents]{}{1}[2]{\section*{Inhaltsverzeichnis}}
\thispagestyle{empty}
\newpage
\pagenumbering{arabic}


\section{Einleitung}

Im Alltag wirken sich viele Einflüsse auf die Verkehrsmittelwahl aus: Die demographische Entwicklung, Infrastrukturmaßnahmen, Veränderungen in den Siedlungsstrukturen, Steuerungsmaßnahmen, veränderte Energiepreise oder Maßnahmen des Mobility Pricings sind nur ein Teil dieser Faktoren. Um eine quantitative Basis zu schaffen, die verkehrsplanerische, betriebswirtschaftliche und politische Entscheidungen unterstützt, werden Verkehrsmodelle eingesetzt.\newline

Ein Modell zur Analyse unterschiedlicher Faktoren auf die Verkehrsnachfrage ist die Discrete-Choice Modellierung. Hier werden jeder Verkehrsalternative Wahrscheinlichkeiten anhand von Nutzenfunktionen zugewiesen, die von unbekannten Parametern abhängen. Diese Parameter werden mittels der Maximum-Likelihood-Methode aus Umfrageergebnissen geschätzt, in denen die Befragten Aussagen über ihr Verhalten bei der Verkehrsmittelwahl machen. Ebenfalls können Parameter mittels der Stated Choice Methode geschätzt werden. Diese basiert auf hypothetischen Entscheidungssituationen zwischen Alternativen. Hierbei bestehen die Alternativen, zwischen denen sich die Befragten entscheiden müssen, aus unterschiedlichen Attributen. %Eine Stated Choice Parameterschätzung beginnt mit Entscheidungen über die Alternativen und ihrer Attribute und der Erwartung möglicher Nutzenfunktionen zur Optimierung des Designs des Stated Choice Experiment. Anschließend wird das Design implementiert und die Befragung umgesetzt. Im letzten Schritt soll das Aufbereiten der Daten und Abschätzen der Parameter erfolgen.
\newline

Der entwickelte Baukasten übernimmt das Einlesen und Aufbereiten der Erhebungsdaten und ermöglicht, durch Berechnungen im Rahmen der Discrete-Choice Modellierung, eine Schätzung und Visualisierung der Parameter für die Verkehrsalternativen. Durch die Möglichkeit, individuelle Nutzenfunktionen und Attributsableitungen zu definieren, soll das Produkt diesen Ablauf vereinfachen und eine automatisierte Parameterschätzung mit Signifikanzen sowie die Visualisierung der Ergebnisse des Modells ermöglichen. 

\subsection{Anmerkungen}
Wunschkriterien außerhalb des Blocks \textbf{Zielbestimmung} werden durch (\textit{WK}) markiert.

\clearpage
\section{Zielbestimmung}
Das Produkt unterstützt Verkehrsingenieure dabei, Discrete-Choice Modelle in der Verkehrsmittelwahl zu erstellen, modifizieren und visualisieren.
\subsection{Musskriterien}
\begin{itemize}
    \item[\textbf{/MK10/}] Der Nutzer kann Erhebungsdaten im CSV-Format importieren.
    \item[\textbf{/MK20/}] Der Nutzer kann Attributsableitungen hinzufügen, ändern, und löschen.
    \newline Folgende Möglichkeiten stehen dem Nutzer zur Verfügung:
    \begin{itemize}[leftmargin=.7in]
        \item[\textbf{/MK21/}] Intervalle
        \item[\textbf{/MK22/}] Gruppen
        \item[\textbf{/MK23/}] Logische Ausdrücke
        \item[\textbf{/MK24/}] Vergleiche
    \end{itemize}
    \item[\textbf{/MK30/}] Der Nutzer kann Alternativen hinzufügen, ändern und löschen.
    \item[\textbf{/MK35/}] Der Nutzer kann für jede Alternative eine Nutzenfunktion definieren und ändern.
    \begin{itemize}[leftmargin=.7in]
        \item[\textbf{/MK36/}] Der Nutzer kann Linearkombinationen für die Nutzenfunktion verwenden.
    \end{itemize}
    \item[\textbf{/MK40/}] Der Nutzer kann Alternativen exportieren und importieren.
    \item[\textbf{/MK45/}] Der Nutzer kann Attributsableitungen exportieren und importieren. 
    \item[\textbf{/MK50/}] Der Nutzer kann die Parameter und Signifikanz aus gegebener Eingabe und Modellstruktur berechnen lassen.
    \item[\textbf{/MK60/}] Das Programm bietet eine Schnittstelle für andere Bibliotheken zur Parameterschätzung und Signifikanzgewinnung. 
    \begin{itemize}
        \item Es wird standardmäßig das Python Paket \emph{Biogeme} verwendet.
    \end{itemize}
    \item[\textbf{/MK70/}] Das Programm visualisiert die gewonnenen Parameter und Signifikanzniveaus.
    \item[\textbf{/MK80/}] Der Nutzer kann die Ergebnisse der Berechnung  exportieren.
    \item[\textbf{/MK90/}] Der Nutzer kann die Tabelle im CSV-Format mit den Attributsableitungen exportieren.
    \item[\textbf{/MK100/}] Der Nutzer kann die Nutzenfunktionen importieren und exportieren.
    \item[\textbf{/MK110/}] Das Programm bietet eine Schnittstelle um die Parameteraufteilung zu erweitern oder einzuteilen anhand der Signifikanzniveaus.
    \item[\textbf{/MK120/}] Der Nutzer kann ein Projekt als Projektdatei speichern und laden.
\end{itemize}

\subsection{Wunschkriterien}
\begin{itemize}
    \item[\textbf{/WK10/}] Das Programm unterstützt arbiträre Funktionen für Attributsableitungen.
    \item[\textbf{/WK20/}] Der Nutzer kann die importierte CSV-Datei graphisch darstellen lassen.
    \item[\textbf{/WK30/}] Der Nutzer wird durch weitere Funktionen in der Bedienbarkeit unterstützt. \newline Folgende Funktionen können dem Nutzer helfen:
    \begin{itemize}[leftmargin=.7in]
        \item[\textbf{/WK31/}] Autovervollständigung
        \item[\textbf{/WK32/}] Typhinweise
        \item[\textbf{/WK33/}] Fehlerunterbindung
    \end{itemize}
    \item[\textbf{/WK40/}] Die Nutzenfunktionen können verschachtelt werden.
    \newline Folgende Möglichkeiten stehen dem Nutzer zur Verfügung:
    \begin{itemize}[leftmargin=.7in]
        \item[\textbf{/WK41/}] Arithmetik: $\beta \cdot (T + (\beta_2 \cdot X_2 \cdots))$
        \item[\textbf{/WK42/}] Exponentialfunktionen: $\log(T)$, ${\rm e}^T$
        \item[\textbf{/WK43/}] Potenzen: $\beta \cdot X^3$
    \end{itemize}
    \item[\textbf{/WK50/}] Das Programm kann um andere Modell-Strukturen erweitert werden.
    \begin{itemize}
        \item Beispielsweise um das \emph{Nested Logit Model}
    \end{itemize}
    \item[\textbf{/WK60/}] Der Nutzer kann die Schwellwerte für die Signifikanz in der Visualisierung konfigurieren.
    \begin{itemize}
        \item Beispielsweise im \emph{Apollo Package}:\\
        $|\text{T-Ratio}| > 1.95$, $|\text{Robust T-Ratio}| > 1.95$
    \end{itemize}    
    \item[\textbf{/WK70/}] Das Programm bietet einen vollwertigen Algorithmus zur Bestimmung von Signifikanzgruppen und Aufteilungen.
    \item[\textbf{/WK80/}] Der Nutzer kann den Dateityp beim Exportieren der Ergebnisse auswählen. 
    \item[\textbf{/WK90/}] Die Benutzeroberfläche soll in englischer Sprache sein.
\end{itemize}

\subsection{Abgrenzungskriterien}
\begin{itemize}
    \item[\textbf{/AK10/}] Das Programm soll nicht auf Mobilgeräten laufen.
\end{itemize}

%\subsection{Abgrenzungskriterien}

\clearpage
\section{Produkteinsatz}
\subsection{Anwendungsbereiche}
Der vorgesehene Anwendungsbereich ist die Verkehrsmodellierung aus Befragungsdaten im Bereich der Verkehrswissenschaften. Die Anwendung eignet sich zur Modellierung für eine Prognose als Basis für verkehrsplanerische, betriebswirtschaftliche und politische Entscheidungen. Ebenso kann sie Verwendung in der Recherche finden.

\subsection{Zielgruppen}
Das Produkt richtet sich an Verkehrsingenieure und Verkehrswissenschaftler. Zur Verwendung sind keine besonderen Vorkenntnisse im technischen Bereich notwendig.\newline 
Es wird eine Open Source Veröffentlichung angestrebt.
  
\subsection{Betriebsbedingungen}
Der Baukasten wird als Desktopanwendung konzipiert, die in Büroumgebung ausgeführt wird. Übliche Betriebsbedingungen sind die Nutzung zu konventionellen Arbeitszeiten. Es ist keine Beobachtung während der Ausführung notwendig, unbeaufsichtigter Betrieb wird unterstützt und die Ausführung kann unterbrochen und zu späteren Zeitpunkten fortgeführt werden. Zur Ausführung der Anwendung wird keine Internetverbindung benötigt.
Mindestanforderungen für die Ausführung der Anwendung ist ein Laptop mit 8 GB RAM, i5 (4 Kerne, 8 logische Kerne) und Betriebssystem Windows 10 oder Windows 11. Ebenfalls soll die Anwendung auf der Workstation mit Windows Server, 8 Kerne (16 logische Kerne) ausgeführt werden können (\textit{WK}).

\clearpage
\section{Produktumgebung}
\subsection{Software}
\begin{itemize}
    \item Die Software wird in Python geschrieben.
    \item Betriebssystem: Windows 10 oder Windows 11
    \item Die Software wird als Open Source veröffentlicht und erweiterbar sein.
\end{itemize}
\subsection{Hardware}
\begin{itemize}
    \item Das Produkt ist eine Desktopanwendung.
    \begin{itemize}
        \item Mindestanforderung: Laptop, 8 GB RAM, Intel Core i5 (4 Kerne, 8 logische Kerne)
        \item (\textit{WK})\textit{: Workstation, Windows Server, 8 Kerne (16 logische Kerne)}
    \end{itemize}
\end{itemize}
\subsection{Schnittstelle}
\begin{itemize}
    \item Die Software nimmt Dateien im CSV-Format entgegen.
    \item Die Anwendung soll über eine Schnittstelle andere Bibliotheken zur Parameterschätzung einbinden können.
    \begin{itemize}
        \item beispielsweise das \emph{Apollo Package}
    \end{itemize}
\end{itemize}

\clearpage
\section{Produktfunktionen}
\subsection{Diagramm}
Das Anwendungsfalldiagramm für alle Produktfunktionen des Programms:
\begin{figure}[H]%
  \centering
  \includegraphics[width=15cm]{specifications/img/use-case/UseCaseDiagramFinal2NWM.jpg}
  \caption{Anwendungsfalldiagramm}
\end{figure} 
\newpage
\startcontents[toc:functionoverview]
\printcontents[toc:functionoverview]{}{3}{\subsection{Funktionsübersicht}}
%\newpage

\subsection{Funktionsbeschreibungen}
\subsubsection*{\textbf{/F10/} Neues Projekt erstellen} \label{sec:f:Neues Projekt erstellen}
\addcontentsline{toc}{subsubsection}{\nameref{sec:f:Neues Projekt erstellen}}
\begin{itemize}
    \item[\underline{Ziel:}] Ein neues Projekt erstellen, in das eine bereits existierende CSV-Datei mit den Erhebungsdaten importiert wird.
    \item[\underline{Vorbedingung:}] keine
    \item[\underline{Beschreibung:}]
    \begin{enumerate}
        \item Nutzer wählt die CSV-Datei aus.
        \item Nutzer speichert gegebenenfalls offenes Projekt.
    \end{enumerate}
    \item[\underline{Erweiterung:}]
    \begin{itemize}
        \item[2a.] Nutzer bricht den Vorgang ab.
    \end{itemize}
    \item[\underline{Kriterien:}] /MK10/
\end{itemize}

\subsubsection*{\textbf{/F11/} Projekt laden} \label{sec:f:Projekt laden}
\addcontentsline{toc}{subsubsection}{\nameref{sec:f:Projekt laden}}
\begin{itemize}
    \item[\underline{Ziel:}] Ein bereits existierendes Projekt laden, um damit weiterarbeiten zu können.
    \item[\underline{Vorbedingung:}] keine
    \item[\underline{Beschreibung:}]
    \begin{enumerate}
        \item Nutzer wählt die Projektdatei aus.
        \item Nutzer speichert gegebenenfalls offenes Projekt.
    \end{enumerate}
    \item[\underline{Erweiterung:}]
    \begin{itemize}
        \item[2a.] Nutzer bricht den Vorgang ab.
    \end{itemize}
    \item[\underline{Kriterien:}] /MK110/
\end{itemize}

\subsubsection*{\textbf{/F12/} Projekt speichern} \label{sec:f:Projekt speichern}
\addcontentsline{toc}{subsubsection}{\nameref{sec:f:Projekt speichern}}
\begin{itemize}
    \item[\underline{Ziel:}] Benutzerdefinierte Attributsableitungen, Nutzenfunktionen, Ergebnisse aus durchgeführten Berechnungen und die originale CSV-Datei sollen in einer Projektdatei gespeichert werden. So kann der Nutzer das Projekt später öffnen und bearbeiten.
    \item[\underline{Vorbedingung:}] Projekt ist geöffnet.
    \item[\underline{Beschreibung:}] 
    \begin{enumerate}
        \item Nutzer speichert das Projekt.
    \end{enumerate}
    \item[\underline{Erweiterung:}]
    \begin{itemize}
        \item[1a.] Nutzer wählt den Speicherort, falls das Projekt zuvor noch nicht gespeichert wurde.
    \end{itemize}
    \item[\underline{Kriterien:}] /MK110/
\end{itemize}

\subsubsection*{\textbf{/F13/} CSV-Datei exportieren} \label{sec:f:CSV-Datei exportieren}
\addcontentsline{toc}{subsubsection}{\nameref{sec:f:CSV-Datei exportieren}}
\begin{itemize}
    \item[\underline{Ziel:}] Die originale CSV-Datei mit hinzugefügten und berechneten Spalten der definierten Attributsableitungen in neuer CSV-Datei exportieren.
    \item[\underline{Vorbedingung:}] Projekt ist geöffnet.
    \item[\underline{Beschreibung:}] 
    \begin{enumerate}
        \item Nutzer exportiert die CSV-Datei.
        \item Nutzer wählt den Speicherort.
    \end{enumerate}
    \item[\underline{Kriterien:}] /MK90/
\end{itemize}

\subsubsection*{\textbf{/F15/} Attributsableitung hinzufügen} \label{sec:f:Attributsableitung hinzufügen}
\addcontentsline{toc}{subsubsection}{\nameref{sec:f:Attributsableitung hinzufügen}}
\begin{itemize}
    \item[\underline{Ziel:}] Eine neue Attributsableitung definieren und hinzufügen. 
    \item[\underline{Vorbedingung:}] Projekt ist geöffnet.
    \item[\underline{Beschreibung:}]
    \begin{enumerate}
        \item Nutzer möchte neue Attributsableitung hinzufügen.
        \item Nutzer gibt den Namen der Attributsableitung ein.
        \item Nutzer gibt die Ableitung ein.
    \end{enumerate}
    \item[\underline{Erweiterung:}]
    \begin{itemize}
        \item[2a.] Besitzt eine andere Attributsableitung bereits den Namen, weist eine Fehler-
        meldung darauf hin. Der Name kann anschließend geändert werden.
        \item[3a.] Beinhaltet die Attributsableitung einen Syntax- oder Typfehler, weist eine Fehlermeldung den Nutzer daraufhin und die Ableitung wird als invalide markiert. 
    \end{itemize}
    \item[\underline{Kriterien:}] /MK20/; /MK21/; /MK22/; /MK23/; /MK24/;\newline/WK10/; /WK30/; /WK31/; /WK32/; /WK33/
\end{itemize}

\subsubsection*{\textbf{/F16/} Attributsableitung ändern} \label{sec:f:Attributsableitung ändern}
\addcontentsline{toc}{subsubsection}{\nameref{sec:f:Attributsableitung ändern}}
\begin{itemize}
    \item[\underline{Ziel:}] Eine bereits existierende Attributsableitung ändern.
    \item[\underline{Vorbedingung:}] Projekt ist geöffnet. Attributsableitung existiert.
    \item[\underline{Beschreibung:}]
    \begin{enumerate}
        \item Nutzer wählt die Attributsableitung aus.
        \item Nutzer ändert die vorherige Attributsableitung.
    \end{enumerate}
    \item[\underline{Erweiterung:}]
    \begin{itemize}
        \item[2a.] 2a. Besitzt eine andere Attributsableitung bereits den Namen, weist eine Fehler-
        meldung darauf hin. Der Name kann anschließend geändert werden.
        \item[2b.] Beinhaltet die neue Ableitung einen Syntax- oder Typfehler, weist eine Fehlermeldung den Nutzer daraufhin und die Ableitung wird als invalide markiert. 
    \end{itemize}
    \item[\underline{Kriterien:}] /MK20/; /MK21/; /MK22/; /MK23/; /MK24/;\newline/WK10/; /WK30/; /WK31/; /WK32/; /WK33/
\end{itemize}

\subsubsection*{\textbf{/F17/} Attributsableitung löschen} \label{sec:f:Attributsableitung löschen}
\addcontentsline{toc}{subsubsection}{\nameref{sec:f:Attributsableitung löschen}}
\begin{itemize}
    \item[\underline{Ziel:}] Eine existierende Attributsableitung löschen.
    \item[\underline{Vorbedingung:}] Projekt ist geöffnet. Attributsableitung existiert.
    \item[\underline{Beschreibung:}]
    \begin{enumerate}
        \item Nutzer löscht die Funktion.
    \end{enumerate}
    \item[\underline{Erweiterung:}]
    \begin{itemize}
        \item[1a.] Wird die Attributsableitung in einer anderen Attributsableitung oder Nutzenfunktion referenziert, weist eine Warnung darauf hin. Der Nutzer kann anschließend fortfahren oder abbrechen.
    \end{itemize}
    \item[\underline{Kriterien:}] /MK20/
\end{itemize}

\subsubsection*{\textbf{/F18/} Attributsableitungen exportieren} \label{sec:f:Attributsableitungen exportieren}
\addcontentsline{toc}{subsubsection}{\nameref{sec:f:Attributsableitungen exportieren}}
\begin{itemize}
    \item[\underline{Ziel:}] Existierende Attributsableitungen exportieren, damit der Nutzer diese in anderen Projekten wiederverwenden kann.
    \item[\underline{Vorbedingung:}] Projekt ist geöffnet. Es existiert mindestens eine Attributsableitung.
    \item[\underline{Beschreibung:}]
    \begin{enumerate}
        \item Nutzer markiert die zu exportierenden Attributsableitungen.
        \item Nutzer exportiert die ausgewählten Ableitungen. 
    \end{enumerate}
    \item[\underline{Kriterien:}] /MK45/
\end{itemize}

\subsubsection*{\textbf{/F19/} Attributsableitungen importieren}
\label{sec:f:Attributsableitungen importieren}
\addcontentsline{toc}{subsubsection}{\nameref{sec:f:Attributsableitungen importieren}}
\begin{itemize}
    \item[\underline{Ziel:}] Attributsableitungen in das geöffnete Projekt importieren. Dies ermöglicht die Wiederverwendung von Attributsableitungen.
    \item[\underline{Vorbedingung:}] Projekt ist geöffnet.
    \item[\underline{Beschreibung:}]
    \begin{enumerate}
        \item Nutzer importiert die Attributsableitungen. 
    \end{enumerate}
    \item[\underline{Erweiterung:}]
    \begin{itemize}
        \item[1a.] Sind invalide Attributsableitungen enthalten, wird der Nutzer mit einer Fehlermeldung darauf hingewiesen und sie werden markiert. 
    \end{itemize}
    \item[\underline{Kriterien:}] /MK45/
\end{itemize}

\subsubsection*{\textbf{/F20/} Alternative hinzufügen} \label{sec:f:Alternative hinzufügen}
\addcontentsline{toc}{subsubsection}{\nameref{sec:f:Alternative hinzufügen}}
\begin{itemize}
    \item[\underline{Ziel:}] Eine neue Alternative definieren und hinzufügen.
    \item[\underline{Vorbedingung:}] Projekt ist geöffnet.
    \item[\underline{Beschreibung:}]
    \begin{enumerate}
        \item Nutzer möchte neue Alternative hinzufügen.
        \item Nutzer gibt den Namen der Alternative ein.
        \item Nutzer gibt die Nutzenfunktion ein.
    \end{enumerate}
    \item[\underline{Erweiterung:}]
    \begin{itemize}
        \item[2a.] Besitzt eine andere Alternative bereits den Namen, weist eine Fehlermeldung darauf hin. Der Name kann anschließend geändert werden.
        \item[3a.] Beinhaltet die Nutzenfunktion einen Syntaxfehler, weist eine Fehlermeldung den Nutzer daraufhin und die Nutzenfunktion wird als invalide markiert. 
    \end{itemize}
    \item[\underline{Kriterien:}] /MK30/; /MK35/; /MK36/; /WK30/; /WK31/; /WK32/; /WK33/
\end{itemize}

\subsubsection*{\textbf{/F21/} Alternative ändern} \label{sec:f:Alternative ändern}
\addcontentsline{toc}{subsubsection}{\nameref{sec:f:Alternative ändern}}
\begin{itemize}
    \item[\underline{Ziel:}] Eine bereits existierende Alternative ändern.
    \item[\underline{Vorbedingung:}] Projekt ist geöffnet. 
    \item[\underline{Beschreibung:}]
    \begin{enumerate}
        \item Nutzer wählt die Alternative aus.
        \item Nutzer ändert die vorherige Alternative.
    \end{enumerate}
    \item[\underline{Erweiterung:}]
    \begin{itemize}
        \item[2a.] Besitzt eine andere Alternative bereits den Namen, weist eine Fehlermeldung darauf hin. Der Name kann anschließend geändert werden.
        \item[2b.] Beinhaltet die Nutzenfunktion einen Syntaxfehler, weist eine Fehlermeldung den Nutzer daraufhin und die Nutzenfunktion wird als invalide markiert. 
    \end{itemize}
    \item[\underline{Kriterien:}] /MK30/; /MK35/; /MK36/; /WK30/; /WK31/; /WK32/; /WK33/
\end{itemize}

\subsubsection*{\textbf{/F22/} Alternative löschen} \label{sec:f:Alternative löschen}
\addcontentsline{toc}{subsubsection}{\nameref{sec:f:Alternative löschen}}
\begin{itemize}
    \item[\underline{Ziel:}] Eine existierende Alternative löschen.
    \item[\underline{Vorbedingung:}] Projekt ist geöffnet. Alternative existiert.
    \item[\underline{Beschreibung:}]
    \begin{enumerate}
        \item Nutzer löscht die Alternative.
    \end{enumerate}
    \item[\underline{Kriterien:}] /MK30/
\end{itemize}

\subsubsection*{\textbf{/F23/} Alternativen exportieren} \label{sec:f:Alternativen exportieren}
\addcontentsline{toc}{subsubsection}{\nameref{sec:f:Alternativen exportieren}}
\begin{itemize}
    \item[\underline{Ziel:}] Existierende Alternativen exportieren, damit der Nutzer diese in anderen Projekten wiederverwenden kann.
    \item[\underline{Vorbedingung:}] Projekt ist geöffnet. Es existiert mindestens eine Alternative.
    \item[\underline{Beschreibung:}]
    \begin{enumerate}
        \item Nutzer markiert die zu exportierenden Alternativen.
        \item Nutzer exportiert die ausgewählten Alternativen. 
    \end{enumerate}
    \item[\underline{Kriterien:}] /MK40/
\end{itemize}

\subsubsection*{\textbf{/F24/} Alternativen importieren}
\label{sec:f:Alternativen importieren}
\addcontentsline{toc}{subsubsection}{\nameref{sec:f:Alternativen importieren}}
\begin{itemize}
    \item[\underline{Ziel:}] Alternativen in das geöffnete Projekt importieren. Dies ermöglicht die Wiederverwendung von Alternativen.
    \item[\underline{Vorbedingung:}] Projekt ist geöffnet.
    \item[\underline{Beschreibung:}]
    \begin{enumerate}
        \item Nutzer importiert die Alternativen. 
    \end{enumerate}
    \item[\underline{Erweiterung:}]
    \begin{itemize}
        \item[1a.] Sind invalide Alternativen enthalten, wird der Nutzer mit einer Fehlermeldung darauf hingewiesen und sie werden markiert. 
    \end{itemize}
    \item[\underline{Kriterien:}] /MK40/
\end{itemize}

\subsubsection*{\textbf{/F25/} Berechnung durchführen} \label{sec:f:Berechnung durchführen}
\addcontentsline{toc}{subsubsection}{\nameref{sec:f:Berechnung durchführen}}
\begin{itemize}
    \item[\underline{Ziel:}] Parameter und zugehörige Signifikanzen berechnen lassen. Die Ergebnisse werden anschließend visualisiert und die Signifikanz der Parameter visuell hervorgehoben.
    \item[\underline{Vorbedingung:}] Projekt ist geöffnet. Es existiert mindestens eine Alternative.
    \item[\underline{Beschreibung:}]
    \begin{enumerate}
        \item Mindestens eine Alternative existiert.
        \item Nutzer lässt die Berechnung durchführen.
        \item Nach der Berechnung werden die Ergebnisse visualisiert.
    \end{enumerate}
    \item[\underline{Kriterien:}] /MK50/; /MK60/; /MK70/ 
\end{itemize} 

\subsubsection*{\textbf{/F28/} Ergebnisse exportieren} \label{sec:f:Ergebnisse exportieren}
\addcontentsline{toc}{subsubsection}{\nameref{sec:f:Ergebnisse exportieren}}
\begin{itemize}
    \item[\underline{Ziel:}] Ergebnisse der Parameter- und Signifikanz-Berechnung exportieren.
    \item[\underline{Vorbedingung:}] Projekt ist geöffnet. Berechnung wurde durchgeführt.
    \item[\underline{Beschreibung:}]
    \begin{enumerate}
        \item Berechnung wird durchgeführt.
        \item Nutzer exportiert die Ergebnisse in einer CSV-Datei.
    \end{enumerate}
    \item[\underline{Erweiterung:}]
    \begin{itemize}
        \item[2a.] \textit{(WK) Der Nutzer kann den Dateityp der Ergebnisse wählen.}
    \end{itemize}
    \item[\underline{Kriterien:}] /MK80/; /WK80/
\end{itemize}

\subsubsection*{\textbf{/F30/} Schwellwerte konfigurieren} \label{sec:f:Schwellwerte konfigurieren}
\addcontentsline{toc}{subsubsection}{\nameref{sec:f:Schwellwerte konfigurieren}}
\begin{itemize}
    \item[\underline{Ziel:}] Schwellwerte in der Visualisierung der Ergebnisse ändern.
    \item[\underline{Vorbedingung:}] Projekt ist geöffnet. Berechnung wurde durchgeführt.
    \item[\underline{Beschreibung:}]
    \begin{enumerate}
        \item Berechnung wird durchgeführt.
        \item Nutzer ändert den Schwellwert.
    \end{enumerate}
    \item[\underline{Kriterien:}] /WK60/
\end{itemize}

\subsubsection*{\textbf{/F31/} Variation konfigurieren} \label{sec:f:Variation konfigurieren}
\addcontentsline{toc}{subsubsection}{\nameref{sec:f:Variation konfigurieren}}
\begin{itemize}
    \item[\underline{Ziel:}] Gewünschte Berechnungsvariationen eingeben
    \item[\underline{Vorbedingung:}] Projekt ist geöffnet. Variationsparameter sind in Spalten verwendet.
    \item[\underline{Beschreibung:}]
    \begin{enumerate}
        \item Nutzer möchte Variation konfigurieren.
        \item Nutzer wählt Verarbeitungsmethode aus.
        \item Nutzer nimmt die notwendigen Konfigurationen vor.
    \end{enumerate}
    \item[\underline{Kriterien:}] /WK70/
\end{itemize}

\subsubsection*{\textbf{/F32/} Modell optimieren}
\label{sec:f:Modell optimieren}
\addcontentsline{toc}{subsubsection}{\nameref{sec:f:Modell optimieren}}
\begin{itemize}
    \item[\underline{Ziel:}] Aus der berechneten Menge von Parametern und Signifikanzen ein verbessertes Modell erstellen und ins Projekt laden.
    \item[\underline{Vorbedingung:}] Projekt ist geöffnet. Variation ist konfiguriert. Berechnung wurde durchgeführt.
    \item[\underline{Beschreibung:}]
    \begin{enumerate}
        \item Berechnung mit Variation wird durchgeführt.
        \item Nutzer lässt Modell optimieren.
        \item Modell wird verbessert.
    \end{enumerate}
    \item[\underline{Kriterien:}] /WK70/
\end{itemize}

\stopcontents[toc:functionoverview]

\clearpage
\section{Produktdaten}
Die Speicherung der Daten erfolgt ausschließlich lokal.
\subsection{Projektdaten}
Bei der Speicherung eines Projekts wird ein projektabhängiger Ordner erstellt, falls keiner vorhanden ist. Die folgenden Daten werden in dem Ordner angelegt oder überschrieben. \newline
/D11/, /D12/ und /D13/ werden automatisch alle 60 Sekunden gespeichert, falls sie bereits als Dateien im Projekt-Ordner vorhanden sind.
/D10/ wird aus Datenschutzgründen nur auf Wunsch des Nutzers gespeichert.
\subsubsection*{\textbf{/D10/} Erhebungsdaten} \label{sec:dd:Erhebungsdaten}
Die CSV-Datei, die dem Projekt zugrunde liegt, wird unverändert gespeichert.
\subsubsection*{\textbf{/D11/} Attributsableitungen} \label{sec:dd:Attributsableitungen}
Die benutzerdefinierten Attributsableitungen werden als Funktionen in einer JSON-Datei gespeichert.
\subsubsection*{\textbf{/D12/} Alternativen und Nutzenfunktionen} \label{sec:dd:Alternativen und Nutzenfunktionen}
Die benutzerdefinierten Alternativen mit jeweiliger Nutzenfunktion werden in einer JSON-Datei gespeichert.
\subsubsection*{\textbf{/D13/} Parameter und Signifikanzen} \label{sec:dd:Parameter und Signifikanzen}
Berechnete Parameter mit jeweiliger Signifikanz werden in einer CSV-Datei gespeichert.
%\subsubsection*{/D70/ CSV-Datei mit Berechnungen}
%Die vollständige CSV-Datei, die sowohl Rohdaten als auch Attribute enthält wird temporär gespeichert.
\subsection{Funktionen}
\subsubsection*{\textbf{/D20/} Attributsableitungen} \label{sec:df:Attributsableitungen}
Die benutzerdefinierten Attributsableitungen können seperat vom Projekt gespeichert werden. Diese können anschließend in andere Projekte geladen werden.
\subsubsection*{\textbf{/D21/} Alternativen und Nutzenfunktionen} \label{sec:df:Alternativen und Nutzenfunktionen}
Die benutzerdefinierten Alternativen mit jeweiliger Nutzenfunktion können seperat vom Projekt gespeichert werden. Diese können anschließend in andere Projekte geladen werden.


\clearpage
\section{Produktleistungen}
\subsection{Benutzbarkeit}
\begin{itemize}
    \item[\textbf{/LB10/}] Die Berechnung der Ergebnisse kann jederzeit abgebrochen werden.
\end{itemize}
\subsection{Zuverlässigkeit}
\begin{itemize}
    \item[\textbf{/LZ10/}] Die Daten /D11/, /D12/ und /D13/ werden automatisch alle 60 Sekunden gespeichert, falls sie bereits als Dateien im Projekt-Ordner vorhanden sind.
\end{itemize}
\subsection{Kommunikation}
\begin{itemize}
    \item[\textbf{/LK10/}] Das Laden einer korrupten CSV- oder Projektdatei wird mit einer Fehlermeldung abgebrochen. Das Programm wird dadurch nicht beeinträchtigt.
    \item[\textbf{/LK20/}] Ungültige Nutzenfunktionen werden visuell und mit Fehlermeldung markiert.
    \item[\textbf{/LK30/}] Ungültige Attributsableitungen werden visuell und mit Fehlermeldung markiert.
    \item[\textbf{/LK40/}] Das Speichern einer Projektdatei mit ungültigen Nutzenfunktionen oder ungültigen Attributsableitungen wird dem Nutzer mit einer Fehlermeldung angemerkt. Danach kann der Nutzer den Speichervorgang fortführen oder abbrechen.
    \item[\textbf{/LK50/}] Das Laden einer Projektdatei mit ungültigen Nutzenfunktionen oder ungültigen Attributsableitungen wird dem Nutzer mit einer Fehlermeldung angemerkt.
    \item[\textbf{/LK60/}] Das Exportieren einer CSV-Datei bei Existenz von ungültigen Attributsableitungen wird dem Nutzer mit einer Fehlermeldung angemerkt. Danach kann der Nutzer den Vorgang fortführen oder abbrechen.
\end{itemize}
\subsection{Effizienz}
\begin{itemize}
    \item[\textbf{/LE10/}] Das Importieren einer CSV-Datei ($\leq$ 500MB) erfolgt innerhalb von maximal 5 Sekunden.
    \item[\textbf{/LE20/}] Das Laden einer Projektdatei ($\leq$ 500MB CSV-Datei, $\leq$ 20 Alternativen, $\leq$ 30 Attributsableitungen) erfolgt innerhalb von maximal 10 Sekunden.
    \item[\textbf{/LE30/}] Die Berechnung der Ergebnisse und die Visualisierung erfolgt innerhalb von maximal 10 Sekunden.
    \item[\textbf{/LE40/}] Das Exportieren der Ergebnisse erfolgt innerhalb von maximal 3 Sekunden.
    \item[\textbf{/LE50/}] Das Exportieren der CSV-Datei mit den Attributsableitungen erfolgt innerhalb von maximal 5 Sekunden.
\end{itemize}

\clearpage
\section{Benutzerschnittstelle}

Die folgenden Bestandteile einer Benutzeroberfläche sind als grober Entwurf zu verstehen. Zusätzlich werden z.~B. noch diverse Dialogfenster umgesetzt, die für das Öffnen und Speichern von Daten erforderlich sind.\\

\subsection{Vertikale Trennlinie im Hauptfenster}
Im Hauptfenster (z.~B. in Abbildung~\ref{gui:fig_columns+model}) soll eine Unterteilung zwischen der Spaltenübersicht (links) und der Konfiguration (rechts) vorgenommen werden. 

Die Verwendung einer Trennlinie in der Mitte des Fensters hat den Vorteil, dass der Überblick über die Spalten erhalten bleibt. Falls ein Abschnitt beim Nutzer nicht erwünscht ist, kann die Unterteilung verändert werden (Abbildung~\ref{gui:fig_splitter1}) oder eine Seite vollständig ausgeblendet werden (Abbildung~\ref{gui:fig_splitter2}).

\begin{figure}[H]%
  \centering
  \includegraphics[width=12cm]{specifications/img/gui-screenshots/splitter1.png}
  \caption{Hauptfenster mit manuell nach links verschobener Trennlinie}
  \label{gui:fig_splitter1}
\end{figure}

\begin{figure}[H]%
  \centering
  \includegraphics[width=12cm]{specifications/img/gui-screenshots/splitter2.png}
  \caption{Hauptfenster mit ausgeblendeten Elementen links}
  \label{gui:fig_splitter2}
\end{figure}

\subsection{Spaltenübersicht}
Die Spaltenübersicht (links in Abbildung~\ref{gui:fig_columns+model}) stellt dem Nutzer alle wichtigen Informationen über die existierenden Spalten zur Verfügung. Hier hat der Nutzer die Möglichkeit, Attributsableitungen zu verwalten (\textbf{/MK20/}). Um selbst definierte und eingelesene Spalten unterscheiden zu können, werden diese verschieden gekennzeichnet. Die dargestellten Spalteninformationen sind zumindest:
\begin{itemize}
    \item Spaltenbezeichner
    \item Datentyp (\emph{str}, \emph{int}, \emph{float}, \emph{bool}, \dots)
    \item Datenquelle (z.~B. aus welcher Datei die Spalte importiert wurde oder wie sie abgeleitet wird (\textbf{/MK21/} bis \textbf{/MK24/})
\end{itemize}

Oberhalb der Spaltendarstellung befindet sich eine Textfeld mit Suchfunktion. Unterhalb können durch die Knöpfe \emph{+}, \emph{-}, \emph{Import} und \emph{Export} die Spalten verwaltet werden (\textbf{/MK20/} und \textbf{/MK45/}).\\

Um später eine automatisierte Variation der Spalten zu ermöglichen (\textbf{/MK110/}), werden hier auch Variablen ohne konkreten Wert erlaubt. Für diese ist ein gesonderter Namensraum zur besseren Unterscheidung vorgesehen. Im Entwurf ist dies mit einem vorangestellten Prozent-Zeichen~(\emph{\%}) demonstriert.


\subsection{Konfiguration}
Im rechten Abschnitt des Hauptfensters (z.~B. in Abbildung~\ref{gui:fig_columns+model}) kann die eigentliche Datenverarbeitung konfiguriert werden.

\subsubsection{Modelldefinition}
\begin{figure}[H]%
  \centering
  \includegraphics[width=12cm]{specifications/img/gui-screenshots/columns+model.png}
  \caption{Hauptfenster mit Spalten- und Modellverwaltung}
  \label{gui:fig_columns+model}
\end{figure}

Im Unterpunkt \emph{Model} kann der Nutzer hier die Nutzenfunktionen zu den gewünschten Alternativen hinterlegen (rechts in Abbildung~\ref{gui:fig_columns+model} (\textbf{/MK35/}). Eine solche Funktion besteht immer aus einem Bezeichner~(\emph{Label}) und der dazugehörigen Definition. Für die Parameter wird ein besonderer Namensraum vorgesehen, damit Parameter jederzeit von anderen Ausdrücken unterschieden werden können. Im Entwurf ist dies mit einem vorangestellten Dollar-Zeichen~(\emph{\$}) demonstriert. Ähnlich wie für Spalten wird dem Nutzer für die Nutzenfunktionen die Möglichkeit gegeben, diese einzeln zu importieren und zu exportieren (\textbf{/MK40/}).//

\subsubsection{Konfiguration der Verarbeitungsmethode}
Unter dem nächsten Unterpunkt \emph{Processing} kann der Nutzer das gewünschte Berechnungsverfahren auswählen (z.~B. eine einfache Maximum-Likelihood-Schätzung wie in Abbildung~\ref{gui:fig_columns+processing} entsprechend \textbf{/MK50/}). Nachdem dieses festgelegt wurde werden die individuellen Optionen des gewählten Verfahrens dargestellt. Bei Bedarf kann der Nutzer diese anpassen (\textbf{/MK60/} und \textbf{/MK110/}).//

\subsubsection{Auswertung}
\begin{figure}[H]%
  \centering
  \includegraphics[width=12cm]{specifications/img/gui-screenshots/columns+evaluation.png}
  \caption{Hauptfenster mit Spaltenverwaltung und Auswertungsdarstellung nach der Berechnung}
  \label{gui:fig_columns+evaluation}
\end{figure}

Im letzten Abschnitt \emph{Evaluation} stößt der Nutzer mit einem Klick auf \emph{Calculate Model} den Berechnungsalgorithmus schlussendlich an (\textbf{/MK50/}). Während der Berechnung wird dem Nutzer grafisch gezeigt, dass eine Berechnung vorgenommen wird.\\

Nach Fertigstellung der Berechnung erfolgt eine Darstellung der geschätzten Parameter und deren Signifikanzen in tabellarischer Form (rechts in Abbildung \ref{gui:fig_columns+evaluation}) (\textbf{/MK70/}). Die vom Nutzer durch Grenzwertsetzung gewünschten Parameter (unter \emph{View Options} definiert) werden entsprechend hervorgehoben (\textbf{/WK60/}). Es besteht am Ende die Möglichkeit durch einen Klick auf \emph{Export Result} das Ergebnis zu exportieren (\textbf{/MK80/}). Die automatisierte Aktualisierung des zuvor definierten Modells (z.~B. durch Übernehmen der signifikantesten Modellvariante) kann durch \emph{Update Model} ausgelöst werden (\textbf{/WK70/}).\\

\newpage

\subsection{Menüleiste}
Die Menüleiste (Abbildungen~\ref{gui:fig_menubar1-file} und \ref{gui:fig_menubar2-edit}) beinhaltet im Entwurf alle benötigen Optionen zum Anlegen, Laden und Speichern eines Projekts und alle gängigen Kurzbefehle (\emph{Widerrufen}, \emph{Wiederholen}, \emph{Ausschneiden}, \emph{Kopieren}, \emph{Einfügen}, \emph{Löschen}, \emph{Suchen}, \emph{Alles auswählen}).

\begin{figure}[H]%
  \centering
  \includegraphics[width=6cm,trim={0 12cm 18cm 0},clip]{specifications/img/gui-screenshots/menubar1-file.png}
  \caption{Hauptfenster mit Menüleistenaufruf \emph{Datei} bzw. \emph{File}}
  \label{gui:fig_menubar1-file}
\end{figure}

\begin{figure}[H]%
  \centering
  \includegraphics[width=6cm,trim={0 8cm 18cm 0},clip]{specifications/img/gui-screenshots/menubar2-edit.png}
  \caption{Hauptfenster mit Menüleistenaufruf \emph{Bearbeiten} bzw. \emph{Edit}}
  \label{gui:fig_menubar2-edit}
\end{figure}

\clearpage
\section{Globale Testfälle}

\subsection{Funktionssequenzen}
Folgende Funktionen sind zu überprüfen.

\begin{table}[H]
\begin{tabularx}{\textwidth}{rX}
\vspace{1mm}
\textbf{/T10/}         & \textbf{Importieren der CSV-Datei} \\ \vspace{1mm}
\textbf{Produktfunktion} & \nameref{sec:f:Projekt laden}\\ \vspace{1mm}
\textbf{Vorbedingung}  & Das Programm ist gestartet. \\ \vspace{1mm}
\textbf{Beschreibung}  & Der Nutzer wählt die CSV-Datei aus. \\
\textbf{Nachbedingung} & Die CSV-Datei wurde importiert und die enthaltenen Werte wurden in das Programm eingelesen.
\end{tabularx}
\end{table}

\begin{table}[H]
\begin{tabularx}{\textwidth}{rX}
\vspace{1mm}
\textbf{/T11/}         & \textbf{Importieren der Nutzenfunktionen (Erfolg)} \\ \vspace{1mm}
\textbf{Produktfunktion} & \nameref{sec:f:Alternativen importieren} \\ \vspace{1mm}
\textbf{Vorbedingung}  & Das Programm ist gestartet. \\ & Die Nutzenfunktionen sind alle valide. \\ & Die verwendeten Spalten existieren alle im aktuellen Datensatz. \\
\vspace{1mm}
\textbf{Beschreibung}  & Der Nutzer wählt die JSON-Dateien aus. \\
\textbf{Nachbedingung} & Die JSON-Dateien wurden importiert und im Programm eingelesen.
\end{tabularx}
\end{table}

\begin{table}[H]
\begin{tabularx}{\textwidth}{rX}
\vspace{1mm}
\textbf{/T12/}         & \textbf{Importieren der Nutzenfunktionen (Teilerfolg)} \\ \vspace{1mm}
\textbf{Produktfunktion} & \nameref{sec:f:Alternativen importieren} \\ \vspace{1mm}
\textbf{Vorbedingung}  & Das Programm ist gestartet. \\ & Die Nutzenfunktionen können invalide sein. \\ & Die verwendeten Spalten können im aktuellen Datensatz fehlen. \\ & Namen der Alternativen können doppelt vorkommen.
\vspace{1mm} \\
\textbf{Beschreibung}  & Der Nutzer wählt die JSON-Dateien aus. \\
\textbf{Nachbedingung} & Die JSON-Dateien wurden importiert und im Programm eingelesen. \\ & Fehlerhafte Angaben werden rot hervorgehoben.
\end{tabularx}
\end{table}

\begin{table}[H]
\begin{tabularx}{\textwidth}{rX}
 \vspace{1mm}
\textbf{/T20/}         & \textbf{Exportieren der CSV-Datei (Erfolg)} \\ \vspace{1mm}
\textbf{Produktfunktion} & \nameref{sec:f:CSV-Datei exportieren}\\
\textbf{Vorbedingung}  & Die CSV-Datei ist im Programm geladen. \\ \vspace{1mm} & Alle Attributsableitungen sind valide. \\ \vspace{1mm}
\textbf{Beschreibung}  & Der Nutzer exportiert die CSV-Datei. \\
\textbf{Nachbedingung} & Es wurde eine neue CSV-Datei mit allen Attributsableitungen erstellt.
\end{tabularx}
\end{table}

\begin{table}[H]
\begin{tabularx}{\textwidth}{rX}
 \vspace{1mm}
\textbf{/T21/}         & \textbf{Exportieren der CSV-Datei (Teilerfolg)} \\ \vspace{1mm}
\textbf{Produktfunktion} & \nameref{sec:f:CSV-Datei exportieren} \\
\textbf{Vorbedingung}  & Die CSV-Datei ist im Programm geladen. \\ \vspace{1mm} & Es existieren invalide Attributsableitungen. \\
\textbf{Beschreibung}  & Der Nutzer exportiert die CSV-Datei. \\ & Das Programm meldet invalide Attributsableitungen. \\ \vspace{1mm} & Der Nutzer führt den Export fort. \\
\textbf{Nachbedingung} & Es wurde eine neue CSV-Datei mit allen validen Attributsableitungen erstellt.
\end{tabularx}
\end{table}

\begin{table}[H]
\begin{tabularx}{\textwidth}{rX}
 \vspace{1mm}
\textbf{/T22/}         & \textbf{Exportieren der CSV-Datei (Abbruch)} \\ \vspace{1mm}
\textbf{Produktfunktion} & \nameref{sec:f:CSV-Datei exportieren} \\
\textbf{Vorbedingung}  & Die CSV-Datei ist im Programm geladen. \\  \vspace{1mm}& Es existieren invalide Attributsableitungen. \\
\textbf{Beschreibung}  & Der Nutzer exportiert die CSV-Datei. \\ & Das Programm meldet invalide Attributsableitungen. \\ \vspace{1mm} & Der Nutzer bricht den Export ab. \\
\textbf{Nachbedingung} & Der Zustand des Programms bleibt unverändert.
\end{tabularx}
\end{table}

\begin{table}[H]
\begin{tabularx}{\textwidth}{rX}
\vspace{1mm}
\textbf{/T23/}         & \textbf{Exportieren der Nutzenfunktion (Erfolg)} \\ \vspace{1mm}
\textbf{Produktfunktion} & \nameref{sec:f:Alternativen exportieren} \\
\textbf{Vorbedingung}  & Die CSV-Datei ist im Programm geladen. \\  \vspace{1mm}& Die Nutzenfunktionen sind alle valide. \\
\textbf{Beschreibung}  & Der Nutzer exportiert die Nutzenfunktion. \\
\textbf{Nachbedingung} & Die Nutzenfunktionen werden als JSON-Dateien exportiert.
\end{tabularx}
\end{table}

\begin{table}[H]
\begin{tabularx}{\textwidth}{rX}
\vspace{1mm}
\textbf{/T24/}         & \textbf{Exportieren der Nutzenfunktion (Teilerfolg)} \\ \vspace{1mm}
\textbf{Produktfunktion} & \nameref{sec:f:Alternativen exportieren} \\
\textbf{Vorbedingung}  & Die CSV-Datei ist im Programm geladen. \\  \vspace{1mm}& Es existieren invalide Nutzenfunktionen. \\
\textbf{Beschreibung}  & Der Nutzer exportiert die Nutzenfunktionen. \\ & Das Programm meldet invalide Nutzenfunktionen. \\ \vspace{1mm} & Der Nutzer führt den Export fort. \\
\textbf{Nachbedingung} & Es wurde für jede Nutzenfunktion eine neue JSON-Datei erstellt.
\end{tabularx}
\end{table}

\begin{table}[H]
\begin{tabularx}{\textwidth}{rX}
\vspace{1mm}
\textbf{/T25/}         & \textbf{Exportieren der Nutzenfunktion (Abbruch)} \\ \vspace{1mm}
\textbf{Produktfunktion} & \nameref{sec:f:Alternativen exportieren} \\
\textbf{Vorbedingung}  & Die CSV-Datei ist im Programm geladen. \\  \vspace{1mm}& Es existieren invalide Nutzenfunktionen. \\
\textbf{Beschreibung}  & Der Nutzer exportiert die Nutzenfunktionen. \\ & Das Programm meldet invalide Nutzenfunktionen. \\ \vspace{1mm} & Der Nutzer bricht den Export ab. \\
\textbf{Nachbedingung} & Der Zustand des Programms bleibt unverändert.
\end{tabularx}
\end{table}

\begin{table}[H]
\begin{tabularx}{\textwidth}{rX} \vspace{1mm}
\textbf{/T26/}         & \textbf{Exportieren einer Ableitungsfunktion}  \\ \vspace{1mm}
\textbf{Produktfunktion} & \nameref{sec:f:Attributsableitungen exportieren} \\
\textbf{Vorbedingung}  & Die CSV-Datei ist im Programm geladen. \\ \vspace{1mm} & Die Berechnungen sind abgeschlossen.   \\ \vspace{1mm}
\textbf{Beschreibung}  & Der Nutzer wählt eine Ableitungsfunktion aus und exportiert diese. \\
\textbf{Nachbedingung} & Es wurde eine neue JSON-Datei erstellt, die die Ableitungsfunktion enthält.
\end{tabularx}
\end{table}

\begin{table}[H]
\begin{tabularx}{\textwidth}{rX} \vspace{1mm}
\textbf{/T27/}         & \textbf{Exportieren mehrerer Ableitungsfunktionen}  \\ \vspace{1mm}
\textbf{Produktfunktion} & \nameref{sec:f:Attributsableitungen exportieren} \\
\textbf{Vorbedingung}  & Die CSV-Datei ist im Programm geladen. \\ \vspace{1mm} & Die Berechnungen sind abgeschlossen.   \\ \vspace{1mm}
\textbf{Beschreibung}  & Der Nutzer wählt mehrere Ableitungsfunktionen aus und exportiert diese. \\
\textbf{Nachbedingung} & Für jede ausgewählte Ableitungsfunktion wurde eine neue JSON-Datei erstellt, die die Ableitungsfunktion enthält.
\end{tabularx}
\end{table}

\begin{table}[H]
\begin{tabularx}{\textwidth}{rX} \vspace{1mm}
\textbf{/T28/}         & \textbf{Exportieren der Ergebnisse}  \\ \vspace{1mm}
\textbf{Produktfunktion} & \nameref{sec:f:Ergebnisse exportieren} \\
\textbf{Vorbedingung}  & Die CSV-Datei ist im Programm geladen. \\ \vspace{1mm} & Die Berechnungen sind abgeschlossen.   \\ \vspace{1mm}
\textbf{Beschreibung}  & Der Nutzer exportiert die berechneten Ergebnisse. \\
\textbf{Nachbedingung} & Es wurde eine neue CSV-Datei erstellt, die die berechneten Ergebnisse enthält.
\end{tabularx}
\end{table}

\begin{table}[H]
\begin{tabularx}{\textwidth}{rX}
 \vspace{1mm}
\textbf{/T30/}         & \textbf{Projekt laden} \\ \vspace{1mm}
\textbf{Produktfunktion} & \nameref{sec:f:Projekt laden} \\ \vspace{1mm}
\textbf{Vorbedingung}  & Das Programm ist gestartet.   \\ \vspace{1mm}
\textbf{Beschreibung}  & Der Nutzer wählt die Projektdatei aus. \\
\textbf{Nachbedingung} & Das Projekt wurde geladen.
\end{tabularx}
\end{table}

\begin{table}[H]
\begin{tabularx}{\textwidth}{rX}
 \vspace{1mm}
\textbf{/T31/}         & \textbf{Projekt speichern (Erfolg)} \\ \vspace{1mm}
\textbf{Produktfunktion} & \nameref{sec:f:Projekt speichern} \\
\textbf{Vorbedingung}  & Das Projekt ist geöffnet.   \\ \vspace{1mm} & Alle Eingaben sind valide. \\ \vspace{1mm}
\textbf{Beschreibung}  & Der Nutzer speichert das Projekt. \\
\textbf{Nachbedingung} & Das Projekt wurde gespeichert.
\end{tabularx}
\end{table}

\begin{table}[H]
\begin{tabularx}{\textwidth}{rX}
 \vspace{1mm}
\textbf{/T32/}         & \textbf{Projekt speichern (Erfolg)} \\ \vspace{1mm}
\textbf{Produktfunktion} & \nameref{sec:f:Projekt speichern} \\
\textbf{Vorbedingung}  & Das Projekt ist geöffnet.   \\ \vspace{1mm} & Es existieren invalide Eingaben. \\
\textbf{Beschreibung}  & Der Nutzer speichert das Projekt. \\ & Das Programm meldet invalide Eingaben. \\ \vspace{1mm} & Der Nutzer führt das Speichern fort. \\
\textbf{Nachbedingung} & Das Projekt wurde gespeichert.
\end{tabularx}
\end{table}

\begin{table}[H]
\begin{tabularx}{\textwidth}{rX}
 \vspace{1mm}
\textbf{/T33/}         & \textbf{Projekt speichern (Abbruch)} \\ \vspace{1mm}
\textbf{Produktfunktion} & \nameref{sec:f:Projekt speichern} \\
\textbf{Vorbedingung}  & Das Projekt ist geöffnet.   \\ \vspace{1mm} & Es existieren invalide Eingaben. \\
\textbf{Beschreibung}  & Der Nutzer speichert das Projekt. \\ & Das Programm meldet invalide Eingaben. \\ \vspace{1mm} & Der Nutzer bricht das Speichern ab. \\
\textbf{Nachbedingung} & Es wurde keine Speicherung vorgenommen.  
\end{tabularx}
\end{table}

\begin{table}[H]
\begin{tabularx}{\textwidth}{rX}
 \vspace{1mm}
\textbf{/T40/}         & \textbf{Attributsableitung hinzufügen (Erfolg)} \\ \vspace{1mm}
\textbf{Produktfunktion} & \nameref{sec:f:Attributsableitung hinzufügen} \\ \vspace{1mm}
\textbf{Vorbedingung}  & Die CSV-Datei wurde vollständig geladen.   \\ \vspace{1mm}
\textbf{Beschreibung}  & Der Nutzer gibt eine valide Attributsableitung ein und fügt sie hinzu. \\
\textbf{Nachbedingung} & Die Attributsableitung wird hinzugefügt.\\
\\ \vspace{1mm} \textbf{/T40.1/}         & \textbf{Intervalle} \\
\textbf{Beschreibung}  & Der Nutzer gibt ein syntaktisch korrektes Intervall ein und fügt es als Ableitungsfunktion hinzu. (MK21) \\
\textbf{/T40.2/}         & \textbf{Gruppen} \\
\textbf{Beschreibung}  & Der Nutzer gibt ein syntaktisch korrekte Gruppe ein und fügt sie als Ableitungsfunktion hinzu. (MK22) \\
\textbf{/T40.3/}         & \textbf{Logische Ausdrücke} \\
\textbf{Beschreibung}  & Der Nutzer gibt einen syntaktisch korrekten logischen Ausdruck ein und fügt ihn als Ableitungsfunktion hinzu. (MK23) \\
\textbf{/T40.4/}         & \textbf{Vergleiche} \\
\textbf{Beschreibung}  & Der Nutzer gibt ein syntaktisch korrekten Vergleich ein und fügt ihn als Ableitungsfunktion hinzu. (MK24) \\
\end{tabularx}
\end{table}

\begin{table}[H]
\begin{tabularx}{\textwidth}{rX}
 \vspace{1mm}
\textbf{/T41/}         & \textbf{Attributsableitung hinzufügen (Misserfolg)} \\ \vspace{1mm}
\textbf{Produktfunktion} & \nameref{sec:f:Attributsableitung hinzufügen} \\ \vspace{1mm}
\textbf{Vorbedingung}  & Die CSV-Datei wurde vollständig geladen.   \\ \vspace{1mm}
\textbf{Beschreibung}  & Der Nutzer gibt eine invalide Attributsableitung ein und fügt sie hinzu. \\
\textbf{Nachbedingung} & Die Attributsableitung wird nicht hinzugefügt. \\ & Der Nutzer erhält eine Fehlermeldung, welche auf den Fehler hinweist.
\end{tabularx}
\end{table}

\begin{table}[H]
\begin{tabularx}{\textwidth}{rX}
 \vspace{1mm}
\textbf{/T42/}         & \textbf{Attributsableitung ändern (Erfolg)} \\ \vspace{1mm}
\textbf{Produktfunktion} & \nameref{sec:f:Attributsableitung ändern} \\
\textbf{Vorbedingung}  & Die CSV-Datei wurde vollständig geladen. \\ \vspace{1mm} & Es existiert mindestens eine Attributsableitung.  \\ \vspace{1mm}
\textbf{Beschreibung}  & Der Nutzer wählt eine Attributsableitung aus, gibt eine valide Attributsableitung ein und wendet die Änderung an. \\
\textbf{Nachbedingung} & Die ausgewählte Attributsableitung wird geändert.
\end{tabularx}
\end{table}

\begin{table}[H]
\begin{tabularx}{\textwidth}{rX}
 \vspace{1mm}
\textbf{/T43/}         & \textbf{Attributsableitung ändern (Misserfolg)} \\ \vspace{1mm}
\textbf{Produktfunktion} & \nameref{sec:f:Attributsableitung ändern} \\
\textbf{Vorbedingung}  & Die CSV-Datei wurde vollständig geladen. \\ \vspace{1mm} & Es existiert mindestens eine Attributsableitung.   \\ \vspace{1mm}
\textbf{Beschreibung}  & Der Nutzer wählt eine Attributsableitung aus, gibt eine valide Attributsableitung ein und wendet die Änderung an. \\
\textbf{Nachbedingung} & Die ausgewählte Attributsableitung wird nicht geändert. \\ & Der Nutzer erhält eine Fehlermeldung, welche auf den Fehler hinweist.
\end{tabularx}
\end{table}

\begin{table}[H]
\begin{tabularx}{\textwidth}{rX}
 \vspace{1mm}
\textbf{/T44/}         & \textbf{Attributsableitung löschen (Erfolg)} \\ \vspace{1mm}
\textbf{Produktfunktion} & \nameref{sec:f:Attributsableitung löschen} \\
\textbf{Vorbedingung}  & Die CSV-Datei wurde vollständig geladen. \\ & Es existiert mindestens eine Attributsableitung.  \\ \vspace{1mm} & Andere Eingaben haben keine Abhängigkeiten zu der ausgewählten Attributsableitung. \\ \vspace{1mm}
\textbf{Beschreibung}  & Der Nutzer wählt eine Attributsableitung aus und löscht sie. \\
\textbf{Nachbedingung} & Die ausgewählte Attributsableitung wird gelöscht.
\end{tabularx}
\end{table}

\begin{table}[H]
\begin{tabularx}{\textwidth}{rX}
 \vspace{1mm}
\textbf{/T45/}         & \textbf{Attributsableitung löschen (Erfolg)} \\ \vspace{1mm}
\textbf{Produktfunktion} & \nameref{sec:f:Attributsableitung löschen} \\
\textbf{Vorbedingung}  & Die CSV-Datei wurde vollständig geladen. \\ & Es existiert mindestens eine Attributsableitung.  \\ \vspace{1mm} & Andere Eingaben haben eine Abhängigkeit zu der ausgewählten Attributsableitung. \\
\textbf{Beschreibung}  & Der Nutzer wählt eine Attributsableitung aus und löscht sie. \\ & Das Programm meldet eine Abhängigkeit zu dieser Ableitung. \\ \vspace{1mm} & Der Nutzer führt das Löschen fort. \\
\textbf{Nachbedingung} & Die ausgewählte Attributsableitung wird gelöscht. \\ & Die abhängigen Eingaben werden als invalide markiert.
\end{tabularx}
\end{table}

\begin{table}[H]
\begin{tabularx}{\textwidth}{rX}
 \vspace{1mm}
\textbf{/T46/}         & \textbf{Attributsableitung löschen (Abbruch)} \\ \vspace{1mm}
\textbf{Produktfunktion} & \nameref{sec:f:Attributsableitung löschen} \\
\textbf{Vorbedingung}  & Die CSV-Datei wurde vollständig geladen. \\ & Es existiert mindestens eine Attributsableitung.  \\ \vspace{1mm} & Andere Eingaben haben eine Abhängigkeit zu der ausgewählten Attrbiutsableitung. \\
\textbf{Beschreibung}  & Der Nutzer wählt eine Attributsableitung aus und löscht sie. \\ & Das Programm meldet eine Abhängigkeit zu dieser Ableitung. \\ \vspace{1mm} & Der Nutzer bricht den Löschvorgang ab. \\
\textbf{Nachbedingung} & Der Zustand des Programms bleibt unverändert.
\end{tabularx}
\end{table}

\begin{table}[H]
\begin{tabularx}{\textwidth}{rX}
 \vspace{1mm}
\textbf{/T50/}         & \textbf{Alternative hinzufügen (Erfolg)} \\ \vspace{1mm}
\textbf{Produktfunktion} & \nameref{sec:f:Alternative hinzufügen} \\ \vspace{1mm}
\textbf{Vorbedingung}  & Die CSV-Datei wurde vollständig geladen.   \\ \vspace{1mm}
\textbf{Beschreibung}  & Der Nutzer gibt eine valide Alternative ein und fügt sie hinzu. \\
\textbf{Nachbedingung} & Die Alternative wurde hinzugefügt.
\end{tabularx}
\end{table}

\begin{table}[H]
\begin{tabularx}{\textwidth}{rX}
 \vspace{1mm}
\textbf{/T51/}         & \textbf{Alternative hinzufügen (Misserfolg)} \\ \vspace{1mm}
\textbf{Produktfunktion} & \nameref{sec:f:Alternative hinzufügen} \\ \vspace{1mm}
\textbf{Vorbedingung}  & Die CSV-Datei wurde vollständig geladen.   \\ \vspace{1mm}
\textbf{Beschreibung}  & Der Nutzer gibt eine Alternative mit invalider Nutzenfunktion oder bereits existierendem Namen ein und fügt sie hinzu. \\
\textbf{Nachbedingung} & Die Alternative wird nicht hinzugefügt. \\ & Der Nutzer erhält eine Fehlermeldung, welche auf den Fehler hinweist.
\end{tabularx}
\end{table}

\begin{table}[H]
\begin{tabularx}{\textwidth}{rX}
 \vspace{1mm}
\textbf{/T52/}         & \textbf{Alternative ändern (Erfolg)} \\ \vspace{1mm}
\textbf{Produktfunktion} & \nameref{sec:f:Alternative ändern} \\ \vspace{1mm}
\textbf{Vorbedingung}  & Die CSV-Datei wurde vollständig geladen. Es existiert mindestens eine Alternative.  \\ \vspace{1mm}
\textbf{Beschreibung}  & Der Nutzer wählt eine Alternative aus, gibt eine valide Alternative ein und wendet die Änderung an. \\
\textbf{Nachbedingung} & Die ausgewählte Alternative wird geändert.
\end{tabularx}
\end{table}

\begin{table}[H]
\begin{tabularx}{\textwidth}{rX}
 \vspace{1mm}
\textbf{/T53/}         & \textbf{Alternative ändern (Misserfolg)} \\ \vspace{1mm}
\textbf{Produktfunktion} & \nameref{sec:f:Alternative ändern} \\ \vspace{1mm}
\textbf{Vorbedingung}  & Die CSV-Datei wurde vollständig geladen. Es existiert mindestens eine Alternative.   \\ \vspace{1mm}
\textbf{Beschreibung}  & Der Nutzer wählt eine Alternative aus, gibt eine Alternative mit invalider Nutzenfunktion oder bereits existierendem Namen ein und wendet die Änderung an. \\
\textbf{Nachbedingung} & Die ausgewählte Nutzenfunktion wird nicht geändert. \\ & Der Nutzer erhält eine Fehlermeldung, welche auf den Fehler hinweist.
\end{tabularx}
\end{table}

\begin{table}[H]
\begin{tabularx}{\textwidth}{rX}
 \vspace{1mm}
\textbf{/T54/}         & \textbf{Alternative löschen} \\ \vspace{1mm}
\textbf{Produktfunktion} & \nameref{sec:f:Alternative löschen} \\ \vspace{1mm}
\textbf{Vorbedingung}  & Die CSV-Datei wurde vollständig geladen. Es existiert mindestens eine Alternative.  \\ \vspace{1mm}
\textbf{Beschreibung}  & Der Nutzer wählt eine Alternative aus und löscht diese. \\
\textbf{Nachbedingung} & Die ausgewählte Alternative wird gelöscht.
\end{tabularx}
\end{table}


\begin{table}[H]
\begin{tabularx}{\textwidth}{rX}
\textbf{/T55/}         & \textbf{Linearkombination von Nutzenfunktionen anwenden} \\ \vspace{1mm}
\textbf{Produktfunktion} & \nameref{sec:f:Alternative hinzufügen}\\ \vspace{1mm}
\textbf{Vorbedingung}  & Die CSV-Datei wurde vollständig geladen. Es existiert mindestens eine Nutzenfunktion.  \\
\textbf{Beschreibung}  & Der Nutzer wählt mindestens eine Nutzenfunktion aus und erstellt eine Linearkombination. (MK41) \\
\textbf{Nachbedingung} & Die konfigurierte Linearkombination der Nutzenfunktionen wird berechnet.
\end{tabularx}
\end{table}


\begin{table}[H]
\begin{tabularx}{\textwidth}{rX} \vspace{1mm}
\textbf{/T60/}         & \textbf{Berechnung durchführen (Erfolg)} \\ \vspace{1mm}
\textbf{Produktfunktion} & \nameref{sec:f:Berechnung durchführen} \\
\textbf{Vorbedingung}  & Die CSV-Datei wurde vollständig geladen. \\ & Es existiert mindestens eine Alternative. \\ \vspace{1mm} & Alle Alternativen sind valide. \\ \vspace{1mm}
\textbf{Beschreibung}  & Der Nutzer führt die Berechnung durch. \\
\textbf{Nachbedingung} & Die Parameter und Signifikanzen aller Alternativen wurden berechnet.
\end{tabularx}
\end{table}

\begin{table}[H]
\begin{tabularx}{\textwidth}{rX} \vspace{1mm}
\textbf{/T61/}         & \textbf{Berechnung durchführen (Erfolg)} \\ \vspace{1mm}
\textbf{Produktfunktion} & \nameref{sec:f:Berechnung durchführen} \\
\textbf{Vorbedingung}  & Die CSV-Datei wurde vollständig geladen. \\ & Es existiert mindestens eine Alternative. \\ \vspace{1mm} & Es existieren invalide Alternativen. \\
\textbf{Beschreibung}  & Der Nutzer führt die Berechnung durch. \\ & Das Programm meldet invalide Alternativen. \\ \vspace{1mm} & Der Nutzer führt die Berechnung fort. \\
\textbf{Nachbedingung} & Die Parameter und Signifikanzen der validen Alternativen wurden berechnet.
\end{tabularx}
\end{table}

\begin{table}[H]
\begin{tabularx}{\textwidth}{rX} \vspace{1mm}
\textbf{/T62/}         & \textbf{Berechnung durchführen (Abbruch)} \\ \vspace{1mm}
\textbf{Produktfunktion} & \nameref{sec:f:Berechnung durchführen} \\
\textbf{Vorbedingung}  & Die CSV-Datei wurde vollständig geladen. \\ & Es existiert mindestens eine Alternative. \\ \vspace{1mm} & Es existieren invalide Alternativen. \\
\textbf{Beschreibung}  & Der Nutzer führt die Berechnung durch. \\ & Das Programm meldet invalide Alternativen. \\ \vspace{1mm} & Der Nutzer bricht die Berechnung ab. \\
\textbf{Nachbedingung} & Der Zustand des Programms bleibt unverändert.
\end{tabularx}
\end{table}

\begin{table}[H]
\begin{tabularx}{\textwidth}{rX} \vspace{1mm}
\textbf{/T63/}         & \textbf{Berechnung durchführen (Abbruch)} \\ \vspace{1mm}
\textbf{Produktfunktion} & \nameref{sec:f:Berechnung durchführen} \\
\textbf{Vorbedingung}  & Die CSV-Datei wurde vollständig geladen. \\ \vspace{1mm} & Es existiert mindestens eine Alternative. \\
\textbf{Beschreibung}  & Der Nutzer führt die Berechnung durch. \\ \vspace{1mm} & Der Nutzer bricht die Berechnung mittendrin ab. \\
\textbf{Nachbedingung} & Der Zustand des Programms bleibt unverändert.
\end{tabularx}
\end{table}



\begin{table}[H]
\begin{tabularx}{\textwidth}{rX} \vspace{1mm}
\textbf{/T70/}         & \textbf{Änderung der Schwellwerte} \\ \vspace{1mm}
\textbf{Produktfunktion} & \nameref{sec:f:Schwellwerte konfigurieren} \\
\textbf{Vorbedingung}  & Die CSV-Datei wurde geladen. \\ & Die Berechnungen sind abgeschlossen. \\ \vspace{1mm} & Die Visualisierung der Ergebnisse wird angezeigt.\\ \vspace{1mm}
\textbf{Beschreibung}  & Der Nutzer ändert den Schwellwert. \\
\textbf{Nachbedingung} & Die Änderung des Schwellwerts wurde übernommen. \\ & Die Visualisierung wird reevaluiert.
\end{tabularx}
\end{table}

\begin{table}[H]
\begin{tabularx}{\textwidth}{rX} \vspace{1mm}
\textbf{/T80/}         & \textbf{Erweiterbarkeit} \\ \vspace{1mm}
\textbf{Produktfunktion} & \nameref{sec:f:Berechnung durchführen} \\
\textbf{Vorbedingung}  & Alle Dateien wurden geladen. Es existieren Angaben zum Alter. Die gewählten Nutzenfunktionen beinhalten das Alter. \\
\vspace{1mm}
\textbf{Beschreibung}  & Bei der Berechnung werden verschiedene Altersgruppen erstellt. \\
\textbf{Nachbedingung} & Es existieren verschiedene Altersgruppen.
\end{tabularx}
\end{table}

\begin{table}[H]
\begin{tabularx}{\textwidth}{rX} \vspace{1mm}
\textbf{/T81/}         & \textbf{Erweiterbarkeit} \\ \vspace{1mm}
\textbf{Produktfunktion} & \nameref{sec:f:Variation konfigurieren} \\
\textbf{Vorbedingung}  & Alle Dateien wurden geladen. Es existiert eine Einteilung der Altersgruppen. \\
\vspace{1mm}
\textbf{Beschreibung}  & Es werden neue Parameter für das Modell eingegeben. \\
\textbf{Nachbedingung} & Es existieren neue Altersgruppen.
\end{tabularx}
\end{table}

\begin{table}[H]
\begin{tabularx}{\textwidth}{rX} \vspace{1mm}
\textbf{/T82/}         & \textbf{Erweiterbarkeit} \\ \vspace{1mm}
\textbf{Produktfunktion} & \nameref{sec:f:Modell optimieren} \\
\textbf{Vorbedingung}  & Alle Dateien wurden geladen. Berechnungen wurden durchgeführt. Es existiert eine Einteilung der Altersgruppen. Es existieren berechnete Signifikanzen. Eine Konfiguration der Parameter ist eingegeben.\\
\vspace{1mm}
\textbf{Beschreibung}  & Das Modell wird optimiert. \\
\textbf{Nachbedingung} & Es existieren neue Altersgruppen. Die Signifikanzen nach der Optimierung sind optimiert oder unverändert im Vergleich zu den Signifikanzen vor der Optimierung.
\end{tabularx}
\end{table}

\subsection{Datenkonsistenzen}
Folgende Datenkonsistenzen sind einzuhalten.
\begin{table}[H]
\begin{tabularx}{\textwidth}{rX}
\textbf{/T1000/}        & Rohdaten werden unverändert gespeichert. \\     \textbf{/T1001/}        & Das Öffnen eines gespeicherten Projektes resultiert in den gleichen Programmdaten wie vor dem Speichervorgang. \\
\end{tabularx}
\end{table}

\clearpage
\section{Qualitätsbestimmung}
\subsection{Funktionalität}
\begin{table}[H]
\centering
\begin{tabular}{lcccc}
\hline
\textbf{Produktqualität} & sehr gut & gut      & normal   & nicht relevant \\ \hline
Angemessenheit           &          &          & $\times$ &                \\
Richtigkeit              & $\times$ &          &          &                \\
Interoperabilität        &          & $\times$ &          &                \\
Ordnungsmäßigkeit        &          &          &          & $\times$       \\
Sicherheit               &          &          & $\times$ &                \\  
\end{tabular}
\end{table}
Die Richtigkeit der Berechnungen nimmt höchsten Stellenwert an. Auch die Interoperabilität, insbesondere die Interaktion mit Biogeme zur Berechnung der Discrete Choice Modelle, ist wichtig. Im Punkt der Sicherheit wird beachtet, dass aus Datenschutzgründen sensible Daten nur auf Verlangen gespeichert werden und Nutzereingaben vor der Ausführung geprüft werden. Ordnungsmäßigkeit spielt keine Rolle in der Funktionalität.

\subsection{Zuverlässigkeit}
\begin{table}[H]
\centering
\begin{tabular}{lcccc}
\hline
\textbf{Produktqualität} & sehr gut & gut      & normal   & nicht relevant \\ \hline
Reife                    &          &          & $\times$ &                \\
Fehlertoleranz           & $\times$ &          &          &                \\
Wiederherstellbarkeit    &          & $\times$ &          &                \\
\end{tabular}
\end{table}
Die Fehlertoleranz nimmt hohen Stellenwert ein um Einschränkungen an den Nutzer gering zu halten. Durch gesicherte Wiederherstellbarkeit der Projektabläufe ist Zuverlässigkeit des Programmes gegeben. 

\subsection{Benutzbarkeit}
\begin{table}[H]
\centering
\begin{tabular}{lcccc}
\hline
\textbf{Produktqualität} & sehr gut & gut      & normal & nicht relevant \\ \hline
Verständlichkeit         &          & $\times$ &        &                \\
Erlernbarkeit            &          & $\times$ &        &                \\
Bedienbarkeit            & $\times$ &          &        &                \\
\end{tabular}
\end{table}
Es wird besonders Wert gelegt auf die Benutzbarkeit des Baukastens durch eine sehr gute Bedienbarkeit. Das Programm soll einfach in der Anwendung sein durch seinen verständlichen und erlernbaren Ablauf.

\subsection{Effizienz}
\begin{table}[H]
\centering
\begin{tabular}{lcccc}
\hline
\textbf{Produktqualität} & sehr gut & gut & normal   & nicht relevant \\ \hline
Zeitverhalten            &          &     & $\times$ &                \\
Verbrauchsverhalten      &          &     & $\times$ &               
\end{tabular}
\end{table}
Die Effizienz sowohl in der Zeit als auch im Speicherbedarf nimmt keinen hohen Stellenwert an. 

\subsection{Änderbarkeit}
\begin{table}[H]
\centering
\begin{tabular}{lcccc}
\hline
\textbf{Produktqualität} & sehr gut & gut      & normal & nicht relevant \\ \hline
Analysierbarkeit         &          & $\times$ &        &                \\
Modifizierbarkeit        & $\times$ &          &        &                \\
Stabilität               &          & $\times$ &        &                \\
Prüfbarkeit              & $\times$ &          &        &                \\
\end{tabular}
\end{table}
Die Änderbarkeit des Programms ist von höchster Bedeutung. Das Programm soll modifizierbar sein, um verschiedene Bibliotheken zur Berechnung von Discrete Choice Modellen zu unterstützen, und die Einbindung von Algorithmen zur Erweiterbarkeit erlauben.

\newpage
% AAAHHH WARUM FUNKTIONIERT DAS NICHT?
% Den gleichen Code in ein neues Dokument kopieren lässt das Glossar wieder erscheinen - aber den Cache leeren ändert hier nichts
\printglossaries

\end{document}